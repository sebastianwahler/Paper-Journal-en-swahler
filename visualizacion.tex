To carry out the visualization of the data set obtained from the previously described intelligent analysis, \texttt{Vis.js} ~\cite{visjsPaginaWeb}, a library or dynamic visualization library based on the Javascript language, was used.
It is designed to be easy to use, to handle large amounts of dynamic data, and to allow manipulation and interaction with the data. The library consists of the DataSet, Timeline, Network, Graph2d, and Graph3d components.

In our particular case we use the "Network" component, which allows us to display networks in graphs. The visualization is easy to use and supports shapes, styles, colors, sizes, images, etc. It works seamlessly in any modern browser for up to a few thousand nodes and edges. To handle a larger number of nodes, Network has clustering support. The grid uses HTML canvas for rendering.

Vis.js provides implementations of Force-directed graph drawing algorithms. These force-directed algorithms attempt to position nodes by considering the forces between two nodes (attractive if connected, repulsive otherwise). They are generally iterative and move nodes one by one until improvement is no longer possible or the maximum number of iterations is reached. The links are more or less the same length and have as few cross links as possible. Connected nodes move closer together while isolated nodes move further to the sides.

%Para llevar a cabo la visualización del conjunto de datos obtenidos del análisis inteligente anteriormente descripto, se utilizó \texttt{Vis.js} ~\cite{visjsPaginaWeb}, una biblioteca o librería de visualización dinámica basada en lenguaje Javascript. 
%La misma está diseñada para que sea fácil de usar, para manejar grandes cantidades de datos dinámicos y para permitir la manipulación y la interacción con los datos. La biblioteca consta de los componentes DataSet, Timeline, Network, Graph2d y Graph3d.

%En nuestro caso particular utilizamos el componente "Network", que permite mostrar redes en grafos. La visualización es fácil de usar y admite formas, estilos, colores, tamaños, imágenes, etc. Funciona sin problemas en cualquier navegador moderno para hasta unos pocos miles de nodos y bordes. Para manejar una mayor cantidad de nodos, Network tiene soporte de agrupamiento. La red utiliza canvas HTML para la renderización.

%Vis.js proporciona implementaciones de algoritmos de diseño forzados "Force-directed graph drawing". Estos algoritmos dirigidos por fuerza intentan posicionar los nodos considerando las fuerzas entre dos nodos (atractivos si están conectados, repulsivos de lo contrario). Generalmente son iterativos y mueven los nodos uno por uno hasta que ya no es posible mejorar o se alcanza el número máximo de iteraciones. Los enlaces tienen más o menos la misma longitud y el menor número posible de enlaces cruzados. Los nodos conectados se juntan más mientras que los nodos aislados se alejan hacia los lados.

