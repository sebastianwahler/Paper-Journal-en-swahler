%Como continuación de este trabajo, existen diversas 
%líneas de investigación que quedan abiertas y en las que es posible continuar trabajando; algunas de ellas, están más directamente relacionadas con este trabajo y son el resultado de cuestiones que han ido surgiendo durante la realización del mismo.

Currently, work continues on the development of the visualization application. It is intended to modify the original PageRank formula, enriching it with additional information according to court records. For example, the possibility of giving a higher initial ranking to those nodes that have a greater weight than others according to the records. It would also be interesting to do the same with the weight of the links between nodes, since a relationship of 2 criminal cases in common between two people is not the same as one of 18 cases in common. In this way we would also be able to give more ranking to those nodes that are related to others, in a greater number of criminal cases. This is certainly relevant to criminal investigation based on criminal records.
It will also seek to delve into various centrality algorithms. There are some notions that are relevant for identifying the importance of a person in the network induced by criminal cases. For example, \textit{betweenness centrality}, which models the extent to which a particular node is between other nodes in a network, or \textit{closeness centrality}, which is the inverse of the sum of shortest paths ( geodesics) that connect a particular node to all other nodes in a network~\cite{newman2005measure}. Similarly, \textit{eigenvector centrality}, is another way of assigning centrality to a network actor based on the idea that if a node has many central neighbors, it should also be central.
%Actualmente se	continúa trabajando en el desarrollo de la aplicación de visualización. Se pretende modificar la fórmula original de PageRank, enriqueciéndola con información adicional según los registros judiciales. Por ejemplo, la posibilidad de darle mayor ranking inicial a aquellos nodos que tengan un peso mayor que otros según los registros. También sería interesante hacer lo mismo con el peso que poseen los enlaces entre nodos, ya que no es lo mismo una relación de 2 casos penales en común entre dos personas, que una de 18 casos en común. De esta manera lograríamos darle más ranking también a aquellos nodos que estén relacionados con otros, en mayor cantidad de casos penales. Esto es sin duda relevante para la investigación criminal basada en antecedentes penales.
%También se buscará profundizar sobre diversos algoritmos de centralidad. Existen algunas nociones que son de relevancia para la identificación de la importancia de una persona en la red inducida por las causas penales. Por ejemplo, \textit{betweenness centrality}, que modela la medida en que un nodo en particular se encuentra entre otros nodos en una red, o \textit{closeness centrality}, que es la inversa de la suma de los caminos más cortos (geodésicas) que conectan un nodo particular con todos los demás nodos de una red~\cite{newman2005measure}. De manera similar, \textit{eigenvector centrality}, es otra forma de asignar la centralidad a un actor de la red basado en la idea de que si un nodo tiene muchos vecinos centrales, también debería ser central.