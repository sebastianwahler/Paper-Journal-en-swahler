Social Network Analysis (SNA) has contributed to criminal investigations and related intelligence activities. A social network models individuals as nodes linked to each other by arcs or edges that represent the relationships between those individuals. These networks, and their properties, are relevant because they represent an abstraction of human relations that allows the highlighting of specific aspects of the ties and individuals~\cite{pm2018practical}~\cite{burcher2020social}. Networks form graph structures, and the properties of these structures represent the properties of social relations. According to Sage~\cite{scott2011sage}, there are four fundamental pillars of network analysis: recognition of the importance of social relationships between individuals, the collection and analysis of data on these relationships, the importance of visual representation of these data, and the need for mathematical and computational models that explain the connection patterns between individuals.

%Desde hace algunos pocos años, el Análisis de Redes Sociales (o SNA por sus siglas en inglés de Social Network Analysis) ha contribuido a las investigaciones criminales y a las actividades de inteligencia relacionadas. Una red social modela individuos como nodos, vinculados entre sí por arcos o aristas que representan las relaciones entre esos individuos. El estudio de estas redes es importante porque se enfoca en la abstracción de las relaciones humanas sobre uno o más aspectos particulares ~\cite{pm2018practical}~\cite{burcher2020social}. De esta manera, las redes conforman estructuras de grafos en las cuales es posible identificar diversas propiedades, tales como la relevancia o la importancia relativa de los nodos individuales en función de las conexiones existentes o el flujo de información. Según Sage~\cite{scott2011sage} , existen cuatro pilares fundamentales del análisis de redes: el reconocimiento de la importancia de las relaciones sociales entre los individuos, la recolección y análisis de datos sobre estas relaciones, la importancia de la representación visual de estos datos y la necesidad de modelos matemáticos y computacionales que expliquen los patrones de conexión entre los individuos.

Several authors have addressed the benefits of studying social networks for criminal investigations. In the mid-1970s, basic models were used to establish and qualify the relationships between individuals or actors in a particular scenario by defining graphs according to the information collected~\cite{harper1975application}. In these cases, the processing was done manually and with several stages of data refinement and evaluation. According to Klerk~\cite{Klerks1999TheNP}, this is the first generation of network analysis in criminalistics. The second generation involved computational tools that automate part of the task of recording and structuring data. These tools also significantly increased the amount of data to be analyzed, making recording and consultation much more agile. The third and current generation establishes the definition of mathematical models and techniques for the generation of new knowledge. Such as the identification of positions of power and influence or the quality of potential witnesses or informants. Metrics like the centrality of a node in a graph are especially useful in this scenario.

%En particular, la vinculación entre el estudio de las redes sociales y la investigación criminal ha sido encarada por varios autores. A mediados de los 70 se utilizaban modelos básicos para establecer y cualificar las relaciones entre individuos o actores de un escenario particular, definiendo grafos de acuerdo a la información recolectada~\cite{harper1975application}, pero el procesamiento era mayoritariamente manual y con varias etapas de refinamiento y valoración de datos. Esta es la que según Klerk~\cite{Klerks1999TheNP} sería la primera generación de análisis de redes en criminalística. La segunda generación involucra el uso de herramientas computacionales que automatiza parte de la tarea de registro y estructuración de datos. Estas herramientas además aumentaron notoriamente la cantidad de datos que se pueden analizar, haciendo mucho más ágil su registro y consulta. La tercera y actual generación establece la definición de modelos y técnicas matemáticas para la generación de nuevo conocimiento, como la identificación de posiciones de poder e influencia o la calidad de potenciales testigos o informantes. Métricas como la centralidad de un nodo en un grafo son especialmente útiles en este escenario.

Krebs~\cite{krebs2002mapping} presented one of the most significant works in this regard; he identified a part of the terrorist network responsible for the attacks in the United States on September 11, 2001. He did it through their social ties with the pilots responsible for the hijacking. The works \cite{medina2014social}, \cite{qin2005analyzing, and \cite{stollenwerk2016taking} have used a similar strategy. On the other hand, the analysis of social networks has also gained interest in traditional criminal investigations such as mafia structures or drug trafficking~\cite{bouchard2013advances}~\cite{bright2015use}~\cite{giommoni2017illicit}~\cite{morselli2009hells}~\cite{morselli2010assessing}. Studies such as Malm's ~\cite{malm2011networks} have made it possible to identify roles in the supply chain of illicit drugs, which entails different criminal risks for each of the collaborators. There are also examples of these strategies applied in other illegal activities, such as art trafficking~\cite{bichler2013small}, money laundry~\cite{colladon2017using}~\cite{soudijn2014using}, police corruption~\cite{lauchs2011corrupt}, and youth gangs~\cite{mcgloin2005policy}. There are also lines of research in the discipline related to cybercrime~\cite{decary2014information}~\cite{decary2012social}~\cite{decary2013reputation}. It is clear then that social network analysis can be applied to a wide range of criminal activities and has been shown to appropriately model characteristics of illegal organizations, assisting in crime prevention and the design of adequate policies to deal with them.

%Uno de los trabajos más importantes al respecto es el de Krebs ~\cite{krebs2002mapping}, en donde se identifica una parte de la red de terroristas que fue responsable de los atentados del 11 de septiembre de 2001 en Nueva York. Aquí identifica agrupaciones de individuos que se conectan entre sí por los pilotos responsables del secuestro de las aeronaves. Otros estudios similares han sido efectivos en consecuencia ~\cite{medina2014social}~\cite{qin2005analyzing}~\cite{stollenwerk2016taking}.  Por otro lado, el análisis de redes sociales ha cobrado también interés en la investigación criminal tradicional como las estructuras de la mafia o el narcotráfico ~\cite{bouchard2013advances}~\cite{bright2015use}~\cite{giommoni2017illicit}~\cite{morselli2009hells}~\cite{morselli2010assessing}. Estudios como el de Malm ~\cite{malm2011networks} han permitido identificar roles en la cadena de suministros para la fabricación de drogas ilícitas, lo que acarrea diferentes riesgos penales para cada uno de los colaboradores. Otros estudios se enfocan en el uso del análisis de las redes sociales para otras actividades criminales, como el tráfico ilícito de arte ~\cite{bichler2013small}, el lavado de dinero ~\cite{colladon2017using}~\cite{soudijn2014using}, corrupción policial ~\cite{lauchs2011corrupt} y bandas juveniles ~\cite{mcgloin2005policy}~\cite{bichler2014magnetic}. Existen también líneas de investigación en la disciplina referente al cibercrimen ~\cite{decary2014information}~\cite{decary2012social}~\cite{decary2013reputation}. Es claro entonces que el análisis de redes sociales puede ser aplicado a un amplio rango de actividades criminales y ha demostrado modelar apropiadamente características propias de las organizaciones ilegales, asistiendo a la prevención del delito y al diseño de políticas adecuadas para enfrentar estas actividades.

However, some difficulties still require intensive studies. The amount of information that is handled is enormous, in many cases with incomplete, contradictory, and, no less frequently, incorrect information. In addition, traditional human relationships are naturally mixed with illicit interactions between individuals, so it is necessary to properly identify their nature and consequences and determine the sensible limits of the analysed social network.

%Existen sin embargo algunas dificultades que requieren aún estudios intensivos. La cantidad de información que debe manejarse es enorme, en muchos casos con información incompleta, contradictoria y no menos frecuentemente incorrecta. Además, las relaciones humanas tradicionales se mezclan naturalmente con las interacciones ilícitas entre los individuos por lo que es necesario identificar apropiadamente su naturaleza y consecuencias y determinar los límites sensatos de la red social analizada. 

Currently, the state agencies in charge of justice and crime prevention have computerized records of the criminal activities detected and information derived from the investigation processes and proceedings. This information essentially constitutes a form of a social network. For our work, we pay special attention to the data produced for this purpose by the police forces of the Province of Chubut and its Judiciary through the Public Prosecutor's Office (MPF~\cite{MPFChubutPaginaWeb}). All this data is registered in a software system called Coirón. This data set contains tens of thousands of records and can be used to model different social networks on which to apply a mathematical and computational analysis. By doing so we can transform this data into information. This will make it possible to learn more about criminal activities and their perpetrators in the jurisdiction of Chubut.

%Actualmente los organismos estatales encargados de la Justicia y la prevención del delito cuentan con registros informatizados de las actividades criminales detectadas, así como de las etapas y eventos del subsecuente proceso penal. Esta información constituye en esencia, una forma de red social. Para este trabajo es de especial interés la información producida a tal efecto por las fuerzas policiales de la Provincia del Chubut y su Poder Judicial de la mano del Ministerio Público Fiscal (MPF ~\cite{MPFChubutPaginaWeb}), registradas en el sistema \texttt{Coirón}. Existen decenas de miles de registros que son utilizados principalmente para la acción penal, pero que pueden ser empleados para modelar diferentes redes sociales sobre las cuales aplicar un análisis matemático y computacional en la búsqueda de nueva información. Esto permitirá conocer más sobre las actividades criminales y sus autores en la jurisdicción de esa provincia, con las particularidades propias de la información registrada digitalmente.

Some tools and techniques can facilitate the analysis and exploration of these large data sets. In this sense, the area of Information Visualization, particularly the Visualization of Large Data Sets, seeks to assist users in such a way~\cite{xu2005criminal}~\cite{feng2019big}~\cite{mathew2021criminal}. It is also important to study the tasks and interactions that the visualization must support since it is these interactions that enable the exploration of information visualization.

%El análisis y exploración de estos grandes conjuntos de datos y sus relaciones debe ser asistido por técnicas y herramientas que faciliten este proceso y reduzcan la carga cognitiva que recae sobre los usuarios. En tal sentido, el área de Visualización de Información, en particular la Visualización de Grandes Conjuntos de Datos, busca asistir a los usuarios de tal manera. La aplicación de técnicas visuales para la representación de este tipo de información no es nueva ~\cite{xu2005criminal}~\cite{feng2019big}~\cite{mathew2021criminal}. También es importante el estudio de las tareas e interacciones que la visualización debe soportar ~\cite{chen2005visualization}, ya que son estas interacciones las que facilitan la exploración de la visualización de información.
