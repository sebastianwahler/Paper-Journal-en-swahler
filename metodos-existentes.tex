Let us remember that in this work our main interest is the assisted identification of criminal gangs and their qualities.
%Recordemos que en este trabajo nuestro principal interés es la identificación asistida de bandas delictivas y sus cualidades.  

People usually move between known places or nodes (home, work, supermarket, restaurant) and along the same streets or routes. The theory suggests that when a crime occurs it is because criminals and victims cross paths within some of these activity zones (node, route). From the analysis of the crime scene, different types of victims and criminals who frequent it can be determined, understand why they go to that place and what makes the criminal-victim duo meet. It is a structured way of knowing and investigating behavior patterns.
%Las personas nos movemos habitualmente entre lugares conocidos o nodos (hogar, trabajo, supermercado, restaurante) y por las mismas calles o rutas. La teoría sugiere que cuando ocurre un delito es porque se cruzan delincuentes y víctimas dentro de algunas de estas zonas de actividad (nodo, ruta). A partir del análisis del lugar del delito se pueden determinar distintos tipos de víctimas y delincuentes que lo frecuentan, entender por qué concurren a ese lugar y qué hace que se encuentre la dupla delincuente-víctima. Es una manera estructurada de conocer e investigar patrones de comportamiento.

On the other hand, it can be deduced that criminals behave the same as the rest of the people, they carry out daily activities, they move along known routes to go from home to work, or to some other place they frequent. That is, they maintain a certain routine in their lives. An offender will tend to commit a crime somewhere that is within or near his daily commute from home to work, from work to a place of recreation, or other usual place.
%Por otro lado se puede deducir que los delincuentes se comportan igual que el resto de las personas, realizan actividades diariamente, se mueven por rutas conocidas para ir de la casa al trabajo, o a algún otro lugar que frecuenten. Es decir, mantienen una cierta rutina en sus vidas. Un delincuente tenderá a cometer un delito en algún lugar que se encuentre dentro o cerca del recorrido que realiza diariamente para trasladarse desde la casa al trabajo, del trabajo a algún lugar de recreación u otro lugar habitual.

Both approaches seek to find the greatest number of occurrence patterns among various events of similar criminality and hourly patterns, as well as the geographical areas where they occur.
%De ambos enfoques se busca encontrar la mayor cantidad de patrones de ocurrencia entre diversos hechos de similar criminalidad y patrones horarios, como así también las zonas geográficas en donde se producen.  

The nature of the ties of the members of a criminal gang is a variable that provides information on the characteristics and similarities of the members of the group, according to specific criteria: family, cultural, proximity ties (they come from the same neighborhood), have shared prison, specialization (criminal skills), experience or other abilities, and other types of bond.
%La naturaleza de los vínculos de los integrantes de una banda delictiva es una variable que aporta información sobre las características y similitudes de los miembros del grupo, atendiendo a criterios concretos: vínculo familiar, cultural, de proximidad (provienen del mismo barrio), han compartido prisión, de especialización (habilidades delictivas), la experiencia u otras capacidades, y otros tipos de vínculo.
