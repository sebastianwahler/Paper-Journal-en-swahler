
Criminal activities in a city or region can range from minor offenses such as theft and robbery to more serious ones such as protection rackets, cybercrime, sexual abuse, and homicides. Law enforcement agencies record these crimes using various methods and technical details. Criminal records typically include information such as the type of crime, date and time of occurrence, location, and identity of the suspect(s) if known.

%The usual criminal activities in a city or region range from minor thefts and robberies to more serious ones such as protection rackets, cybercrime, sexual abuse and homicides. All of them, when detected, are recorded in different ways by law enforcement, with varied technical details. Criminal records usually include the type of crime, date and time and localization, and of course the corresponding perpetrators whenever known.

This information supports the judicial investigation processes of each case, but over time they constitute an extensive knowledge base on which it is possible to extract valuable information for crime prevention and the search for justice. For example, it is possible to identify relationships between people based on a transitive analysis of criminal events in time and space that suggest the formation of either formal or informal criminal gangs. Relations such as the friendship between various delinquents can also be inferred from criminal records. This social ties are of the utmost importance for crime prevention, as well for resolution of unfinished cases.

%This information supports the judicial investigation processes of each case, but over time they constitute an extensive knowledge base on which it is possible to extract valuable information for crime prevention and the search for justice. For example, it is possible to identify relationships between people based on a transitive analysis of criminal events in time and space that suggest the formation of either formal or informal criminal gangs. The relations of friendship or convenience between various perpetrators of criminal activities can also be inferred from criminal records and is extremely relevant for crime prevention and the resolution of unfinished cases.

Criminal organizations are groups that operate outside the law. They carry out illegal activities for their benefit and to the detriment of other individuals or social groups~\cite{finckenauer2005problems}. These groups can be of different sizes and cover varied geographic areas. Frequently these groups conflict with each other. One of the particular characteristics of this type of organization is the anonymity and discretion of their members, being protective of each other. Criminals know that they are part of a network where the anonymity of each member depends heavily on the anonymity of the rest. This demonstrates the importance of collecting and analyzing information associated with the social ties of criminals.

%Criminal organizations are groups that operate outside the law, carrying out illegal activities for their own benefit and to the detriment of other individuals or social groups~\cite{finckenauer2005problems}. They can be of different sizes and cover varied geographical areas which are, in many cases, under conflict with other similar groups. One of the particular characteristics of this type of organization is the \textit{anonymity} and/or discretion of their members, being protective to each other. This is of vital importance for criminals as they are regularly involved on illegal activities constantly pursued by public security agencies.They know they are part of a complex puzzle where some missing pieces, when discovered, may inevitably expose other members. This requires studies of existing information in order to identify criminals and take appropriate action to prevent crime.

In many cases, the criminal acts are perpetrated by individuals of low rank in the group, with little responsibility and motivated by immediate reward, aspirations of promotion, and higher reputation in their circle of contacts. On the other hand, the masterminds are individuals of higher positions in the hierarchy, with more responsibility in the criminal organization. Those individuals have leadership qualities, long-term interests, and a constant concern for retaining power for personal benefit. Security agencies usually have a record of the perpetrators, while the masterminds are more strenuous to identify. Additionally, the hierarchical structures of the gang, the way they operate, and the inherent culture of the socio-economic class lead to an entangled set of inner codes that generates more obstacles for the identification of the organization as a whole. In this context, we believe that the research and development presented in this article are a significant contribution to the prevention and resolution of criminal activities.

%The task is not easy, since a gang is formed by members of varied degrees of commitment. In many cases, the criminal acts that are evident in society are perpetrated by individuals of low rank in the group, with low responsibility and motivated by immediate reward, aspirations of promotion and higher reputation in their own circle of contacts. On the other hand, there are other individuals of higher positions in the hierarchy, with more responsibility in the criminal organization, who have leadership qualities, long-term interests, and a constant concern for the conservation of power for personal benefit. Frequently, the individuals of the first group are the ones who commit crimes detected and registered by security agencies, while the members of the second group are kept \textit{under the radar} most of the time.Additionally, the hierarchical structures of the gang, the way they operate, and the inherent culture of the socio-economic class lead to an entangled set of inner codes that make it difficult to identify the criminal organization as a whole, with its members and related activities. This is where our work can play a significant role.

We have worked within two areas of Computer Science; Information Visualization, particularly Visualization of Large Data Sets, which  translate information into a visual context~\cite{xu2005criminal}~\cite{feng2019big}~\cite{mathew2021criminal}, such as a map or graph, to make data easier for the human brain to understand and pull insights from~\cite{chen2005visualization}, and the Social  Networks Analysis, which investigate social structures through the use of networks and graph theory.

%In order to identify criminal gangs, their members and the degree of importance of each one of them, two areas of Computer Science are of interest: the area of Information Visualization, particularly \textit{Visualization of Large Data Sets}, which seeks to assist users in the proper understanding of the information, and the \textit{Analysis of Social Networks}, where techniques and formalisms are used to understand the structures of the networks and their nodes. The application of visual techniques for the representation of this type of information is important ~\cite{xu2005criminal}~\cite{feng2019big}~\cite{mathew2021criminal}, as well as the study of the tasks and interactions that visualization must support ~\ cite{chen2005visualization}.

In this line of research, we study the application of these techniques to a real scenario, using criminal records of the Department of Justice\footnote{Ministerio Publico Fiscal} of the province of Chubut in Argentina. We developed active software components for the visualization and intelligent analysis of data, incorporating notions of graph analytics. In particular, in this work we are interested in considering the \textit{PageRank} algorithm, contributing to the detection of relevant criminals among \textit{communities of individuals}, in a similar way it is applied to rank web pages. In order to do this, we use real records of criminal activities through the collaboration of the Public Prosecutor's Office. 

%In this line of research we study the application of these techniques to a real scenario, using criminal records of the Department of Justice\footnote{Ministerio Publico Fiscal} of the province of Chubut in Argentina. We developed active software components for the visualization and intelligent analysis of data, incorporating notions of graph analytics. In particular, in this work we are interested in considering the \textit{PageRank} algorithm, contributing to the detection of relevant criminals among \textit{communities of individuals}, in the similar way it is applied to rank web pages. In order to do this, we use real records of criminal activities through the collaboration of the Public Prosecutor's Office. 
%, which provides the knowledge base to induce the relationship graph as detailed in the following section.

%Dejo este párrafo de ejemplo. Al terminar el artículo habría que actualizarlo
The rest of the article is structured as follows. The next section reviews the state-of-the-art in terms of visualization testing. In the subsequent sections, we continue with the presentation of the black-box and white-box testing tools for information visualizations.
We develop a case study to illustrate both kinds of testing. The case study is based on a C\# tool designed for the visualization of geological data, and it exemplifies the process of finding errors with tools and methods presented in this work.
The last section presents the reached conclusions and the intended future work.
