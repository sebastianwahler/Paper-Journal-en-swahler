
En la actualidad las actividades criminales habituales en una ciudad o región van desde hurtos y robos de poca importancia, hasta otros de mayor gravedad como amenazas, cibercrimen, abusos sexuales y  homicidios. Todos ellos son registrados de diferentes formas por las fuerzas de la ley, con datos de variada precisión que incluyen usualmente la tipificación del delito, los datos en tiempo y espacio, y en muchas ocasiones los autores correspondientes.

Toda esta información respalda los procesos de investigación judicial de cada caso, pero con el transcurso del tiempo constituyen una extensa base de conocimiento sobre la cual es posible extraer valiosa información para la prevención del delito y la búsqueda de la justicia. Por ejemplo, es posible identificar relaciones entre personas de acuerdo a un análisis transitivo de eventos criminales en tiempo y espacio que sugieren la conformación de bandas delictivas. Las relaciones de amistad o conveniencia entre diversos autores de actividades criminales también puede inferirse de los registros delictivos y es de extrema relevancia para la prevención del delito y la resolución de casos inconclusos.

Las organizaciones criminales son grupos que operan fuera de la ley, realizando actividades ilegales en beneficio propio y en detrimento de otros individuos o grupos sociales~\cite{finckenauer2005problems}. Pueden ser de diverso tamaño y cubrir áreas geográficas variadas, en muchos casos en conflicto con otras organizaciones similares. Una de las características particulares de este tipo de organizaciones es que, al estar enfocadas en actividades ilegales perseguidas por los organismos de seguridad pública, el anonimato y/o la discreción de sus miembros es de vital importancia. Esto requiere estudios de la información existente con el fin de identificar los criminales y realizar acciones apropiadas para la prevención del delito.

Los miembros de las organizaciones criminales tienen a su vez diversos grados de compromiso con cada una de ellas. En muchos casos los hechos criminales que son evidentes en la sociedad ocurren por individuos de baja jerarquía y responsabilidad en el grupo, motivados por la recompensa inmediata, las aspiraciones de ascenso y la reputación en su propio círculo de contactos. Por otro lado, existen otros individuos de mayor jerarquía y responsabilidad en la organización criminal, que ostentan cualidades de liderazgo, intereses a largo plazo, y la constante preocupación por la conservación del poder para el beneficio personal y de la organización. Con frecuencia, son los individuos del primer grupo los que cometen delitos percibidos y registrados por las fuerzas policiales, mientras que los miembros del segundo grupo se mantienen con mayor discreción. Adicionalmente, las estructuras jerárquicas, la forma de operar, y la cultura inherente de sus realidades socio-económicas  imponen códigos propios que hacen difícil la identificación de la organización delictiva como un todo, con sus miembros y actividades relacionadas. Es aquí donde nuestro trabajo puede aportar un rol significativo.

En tal sentido, con el objetivo de ayudar en la identificación de las bandas delictivas, sus integrantes y el grado de importancia de cada uno dentro de ellas, son de interés dos áreas de las Ciencias de la Computación: el área de Visualización de Información, en particular la Visualización de Grandes Conjuntos de Datos, que busca asistir a los usuarios en la adecuada comprensión de la información, y el Análisis de Redes Sociales, en donde se emplean técnicas y formalismos para la comprensión de las estructuras de las redes y sus nodos. 
La aplicación de técnicas visuales para la representación de este tipo de información es importante ~\cite{xu2005criminal}~\cite{feng2019big}~\cite{mathew2021criminal}, así como el estudio de las tareas e interacciones que la visualización debe soportar ~\cite{chen2005visualization}, ya que son estas interacciones las que facilitan la exploración de la visualización de información.

En esta línea de investigación estudiamos la aplicación de estas técnicas y tecnologías, contribuyendo además al desarrollo de componentes de software para la visualización y análisis inteligente de los datos, incorporando nociones de analítica de grafos. 
En particular en este trabajo nos interesa la consideración del algoritmo de \textit{PageRank}, aportando a la detección de delincuentes de relevancia entre las \textit{comunidades de individuos}, de la misma forma que se puede discriminar la importancia de un conjunto de páginas web a través de sus vínculos. 
Para esto se cuenta con los registros de actividades criminales a través de la colaboración del Ministerio Público Fiscal de la provincia del Chubut, que provee la base de conocimientos para inducir el grafo de relaciones como se detalla en la siguiente sección.